\chapwithtoc{Conclusion}
\section{Our Contribution}
In this work, we proposed an extensive dataset \textbf{Czech News} dataset with a large amount of metadata,
created with our \textbf{C'monCrawl} utility. We have then evaluated mostly
transformer fine-tuning approaches
and created a strong baseline for all 4 tasks. We found all our models
beating the human baseline on all tasks, with the biggest difference of 44\% on the Server task.
Lastly, we have in-depth analyzed the performance of the models on the tasks
and showed their weaknesses.

Due to copyright issues, we can't redistribute the dataset, but we offer a instructions
for manual download~(\autoref{chap:download}).
We also open-source the code for C'monCrawl\footnote{\url{https://github.com/hynky1999/Rocnikovy-Projekt}}.
Furthermore, we share all of our final models at \ac{hf} hub.
\begin{itemize}
    \item \textbf{Server}: \url{https://huggingface.co/hynky/Server}
    \item \textbf{Category}: \url{https://huggingface.co/hynky/Category}
    \item \textbf{Gender}: \url{https://huggingface.co/hynky/Gender}
    \item \textbf{Day of week}: \url{https://huggingface.co/hynky/Day_of_week}
\end{itemize}
We also create a user-friendly application~(see \autoref{chap:gradio-app})
for all of the tasks.

\section{Future Work}
When it comes to the dataset itself we have discussed the problematic crawling of 
Novinky.cz~(\autoref{sec:server-desc}). We have also
pointed out the category task fallbacks~(\autoref{sec:final-model-performance-on-category}).
which could be improved. Lastly, novel news server additions to the dataset would be 
welcome.

When it comes to tasks themselves, we haven't dealt with transformers issues as 
aligned in \autoref{sec:benefits-and-drawbacks}. We thus encourage researches
to apply memory-efficient transformers to the tasks. Further, it would be 
interesting to include more features when classifying.
Lastly due to the extensiveness of the dataset, other Classification or Regression tasks
could be researched, for example, the Number of comments in discussion section.






