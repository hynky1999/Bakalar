
\chapwithtoc{Introduction}

Introduction should answer the following questions, ideally in this order:
\begin{enumerate}
\item What is the nature of the problem the thesis is addressing?
\item What is the common approach for solving that problem now?
\item How this thesis approaches the problem?
\item What are the results? Did something improve?
\item What can the reader expect in the individual chapters of the thesis?
\end{enumerate}

Expected length of the introduction is between 1--4 pages. Longer introductions may require sub-sectioning with appropriate headings --- use \texttt{\textbackslash{}section*} to avoid numbering (with section names like `Motivation' and `Related work'), but try to avoid lengthy discussion of anything specific. Any ``real science'' (definitions, theorems, methods, data) should go into other chapters.
\todo{You may notice that this paragraph briefly shows different ``types'' of `quotes' in TeX, and the usage difference between a hyphen (-), en-dash (--) and em-dash (---).}

It is very advisable to skim through a book about scientific English writing before starting the thesis. I can recommend `\citetitle{glasman2010science}' by \citet{glasman2010science}.
