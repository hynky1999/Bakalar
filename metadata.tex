

%%% Choose a language %%%

\newif\ifEN
\ENtrue   % uncomment this for english
%\ENfalse   % uncomment this for czech

%%% Configuration of the title page %%%

\def\ThesisTitleStyle{mff} % MFF style
%\def\ThesisTitleStyle{cuni} % uncomment for old-style with cuni.cz logo
%\def\ThesisTitleStyle{natur} % uncomment for nature faculty logo

\def\UKFaculty{Faculty of Mathematics and Physics}
%\def\UKFaculty{Faculty of Science}

\def\UKName{Charles University in Prague} % this is not used in the "mff" style

% Thesis type names, as used in several places in the title
\def\ThesisTypeTitle{\ifEN BACHELOR THESIS \else BAKALÁŘSKÁ PRÁCE \fi}
%\def\ThesisTypeTitle{\ifEN MASTER THESIS \else DIPLOMOVÁ PRÁCE \fi}
%\def\ThesisTypeTitle{\ifEN RIGOROUS THESIS \else RIGORÓZNÍ PRÁCE \fi}
%\def\ThesisTypeTitle{\ifEN DOCTORAL THESIS \else DISERTAČNÍ PRÁCE \fi}
\def\ThesisGenitive{\ifEN bachelor \else bakalářské \fi}
%\def\ThesisGenitive{\ifEN master \else diplomové \fi}
%\def\ThesisGenitive{\ifEN rigorous \else rigorózní \fi}
%\def\ThesisGenitive{\ifEN doctoral \else disertační \fi}
\def\ThesisAccusative{\ifEN bachelor \else bakalářskou \fi}
%\def\ThesisAccusative{\ifEN master \else diplomovou \fi}
%\def\ThesisAccusative{\ifEN rigorous \else rigorózní \fi}
%\def\ThesisAccusative{\ifEN doctoral \else disertační \fi}



%%% Fill in your details %%%

% (Note: \xxx is a "ToDo label" which makes the unfilled visible. Remove it.)
\def\ThesisTitle{Implicit Information Extraction from News Stories}
\def\ThesisAuthor{Hynek Kydlíček}
\def\YearSubmitted{2023}

% department assigned to the thesis
\def\Department{Institute of Formal and Applied Linguistics}
% Is it a department (katedra), or an institute (ústav)?
\def\DeptType{Institute}

\def\Supervisor{Mgr. Jindřich Libovický, Ph.D.}
\def\SupervisorsDepartment{Institute of Formal and Applied Linguistics}

% Study programme and specialization
\def\StudyProgramme{Computer Science}
\def\StudyBranch{Artificial Intelligence Bc.}

\def\Dedication{%
Firstly, I would like to thank my supervisor Mgr. Jindřich Libovický, Ph.D. for his guidance and support throughout the whole process.
Secondly, I am grateful to the faculty for providing me with computational resources.
Last but not least, I would like to extend my thanks to my family, girlfriend, and friends for their support and encouragement.
}

\def\AbstractEN{%
This work deals with information extraction from Czech News Stories. We focus on four tasks: Publishing server,
Article category, Author's textual gender and Publication day of week.
Due to the absence of a suitable dataset for the tasks, we present CZEch NEws Classification dataset (CZE-NEC),
one of the most extensive Czech classification datasets, composed of news articles from
various sources, spanning over twenty years. Tasks are solved using
Logistic Regression and pre-trained Transformer encoders.
Emphasis is put on fine-tuning methods of the Transformer models, which are evaluated in detail.
The models are compared to human evaluators, revealing significant 
superiority over humans on all tasks. Furthermore, the models are pitted against
the commercial large language model GPT-3, outperforming it on half of the tasks,
despite GPT-3 being significantly larger.
Our work sets strong baseline results on CZE-NEC allowing for further research in the field.
}

\def\AbstractCS{%
Tato práce se zabývá extrakcí informací z českých zpravodajských článků. Zaměřujeme se na čtyři úlohy:
vydavatelský server, kategorie článku, textový gender autora a den vydání článk.
Vzhledem k absenci vhodné datové sady pro tyto úlohy představujeme datovou sadu CZEch NEws Classification (CZE-NEC),
jeden z největších českých klasifikačních datasetů, který je složen ze zpravodajských článků z různých zdrojů
pokrývající období dvaceti let. Úlohy jsou řešeny pomocí Lineární regrese a předtrénovaných Transformerů.
Důraz je kladen na metody dotrénování Transformerů, které jsou podrobně vyhodnoceny. Modely jsou porovnány s lidskými
hodnotiteli, kteří zaostávají za modely na všech úlohách. Dále jsou modely porovnány s komerčním velkým jazykovým modelem GPT-3,
který je překonán na polovině úloh, přestože je GPT-3 výrazně větší.
Naše práce představuje silný startovní výsledek na sadě CZE-NEC, který umožňuje další výzkum v této oblasti.
}

% 3 to 5 keywords (recommended), each enclosed in curly braces.
% Keywords are useful for indexing and searching for the theses by topic.
\def\Keywords{%
{Information Extraction}, {Czech News Classification Dataset}, {Transformer}
}

% If your abstracts are long and do not fit in the infopage, you can make the
% fonts a bit smaller by this setting. (Also, you should try to compress your abstract more.)
% Alternatively, consider increasing the size of the page by uncommenting the
% geometry modification in thesis.tex.
\def\InfoPageFont{}
%\def\InfoPageFont{\small}  %uncomment to decrease font size

\ifEN\relax\else
% If you are writing a czech thesis, you additionally need to fill in the
% english translation of the metadata here!
\def\ThesisTitleEN{\xxx{Thesis title in English}}
\def\DepartmentEN{\xxx{Name of the department in English}}
\def\DeptTypeEN{\xxx{Department}}
\def\SupervisorsDepartmentEN{\xxx{Superdepartment}}
\def\StudyProgrammeEN{\xxx{study programme}}
\def\StudyBranchEN{\xxx{study branch}}
\def\KeywordsEN{%
\xxx{{key} {words}}
}
\fi
